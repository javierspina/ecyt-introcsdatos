\documentclass{beamer}


% Paquetes
\usepackage[T1]{fontenc}    % Para escribir en español
\usepackage{hyperref}    % Para hacer links a la web
\usepackage{listings}    % Para incluir código


% Estilos
\usetheme{Luebeck}
\usecolortheme{seahorse}


% Info para la title page
\title{Concurrencia a Museos de Argentina}
\subtitle{Relación con otros consumos culturales}
\author{Nazareno Magallanes \and Javier Spina \and Lautaro Terreno}
\institute[ECyT]
{
  Escuela de Ciencia y Tecnología
  \and
  Universidad Nacional de San Martín
}
\date{Noviembre 2022}


% Inicio del documento
\begin{document}

\frame{\titlepage}

\begin{frame}
\frametitle{Presentación del \textit{dataset}}

\begin{itemize}
\item<1-> Se trabajó con el \textit{dataset} asociado a la \textbf{Encuesta Nacional de Consumos Culturales} realizada en 2017. Son datos abiertos del Ministerio de Cultura de la Nación, disponible en \href{https://datos.cultura.gob.ar/dataset/encuesta-nacional-de-consumos-culturales-2017}{su página web.}
\item<2-> Además del archivo csv, se contó con el cuestionario aplicado y el Informe General realizado por el mismo Ministerio.
\item<3->El \textit{dataset} consta de 2802 unidades de análisis y 450 variables. Cada una de las variables se corresponde con las preguntas del cuestionario asociado.
\item <4->Cada unidad de análisis es una persona encuestada.
\end{itemize}

\end{frame}

\begin{frame}[fragile]
\frametitle{Un primer vistazo}

\begin{itemize}
\item<1->Al comenzar a trabajar con el \textit{dataset}, se encontraron diversas tareas para hacer antes de hacer una manipulación concreta de los datos. Fundamentalmente, los nombres de las variables: 

\begin{lstlisting}
> colnames(encuesta)
[1] "id"  "pondera_dem"  "fecha"  "region"
[5] "sexo"  "edad"  "p1"  "p1otros"
[9] "p2"  "p2_1"  "p2_1otro"  "p2_2"
[13] "p3"  "p4"  "p5"  "p6horas"
\end{lstlisting}

\item<2->La gran mayoría de las variables es de la forma p + número de pregunta del cuestionario.
\end{itemize}

\end{frame}

\begin{frame}
\frametitle{Limpieza de datos}

\begin{itemize}
\item<1-> Para empezar, se transformaron los nombres de las variables considerando la pregunta asociada del cuestionario. Por ejemplo, p98 se renombró a concurre\_museo porque se corresponde a la pregunta 98: ¿Concurrió a algún museo durante el último año?
\item<2-> Se debieron eliminar 2 unidades de análisis que estaban mal cargadas.
\end{itemize}

\end{frame}

\begin{frame}
\frametitle{Búsqueda de la problemática}

\begin{itemize}
\item<1-> Ahora comenzaba la trayectoria para buscar una problemática a resolver mediante la Ciencia de Datos. Se comenzó usando un escenario hipótetico para segmentar consumidores de una bebida energética y sugerir estrategias basadas en los consumos culturales.
\item<2-> Luego, se intentó centrar el problema en buscar una diferencia de preferencias culturales entre nativos digitales y no-nativos digitales.
\item<3-> Mediante varias iteraciones de lectura, interpretación, pruebas y errores se vió que el mismo informe mostraba una problematica mostrada por la Encuesta: hay una tendencia negativa en la concurrencia a los museos argentinos en determinados sectores sociales.
\end{itemize}

\end{frame}

\begin{frame}
\frametitle{La pregunta y su contexto}

\begin{block}{La pregunta}
¿Cómo aumentar la concurrencia a los museos?
\end{block}

\begin{block}{Contexto}

Poner acá primeros gráficos parecidos a los de la página 23 del informe para mostrar contexto. Quizá alguno más puede aportar, hay que ver cuál.
Quizá apoyarnos con el dataset extra de ubicaciones de los museos en el país, considerar si el factor de que la gente no tiene ni 1 auto en el nivel socio económico q no está yendo a museo.

\end{block}

\end{frame}

\begin{frame}
\frametitle{Estrategia para responder la pregunta}

Acá hablar de que buscamos de los que SI fueron a museo tuvieron ciertos preferencias en otros consumos culturales como música, tv, películas. 
Teniendo en cuenta esa segmentación, buscar a la gente que tiene esas preferencias y NO fue a museos y proponer esto como estrategia para alentar la concurrencia, invocando el dato de q 75 por ciento  de la gente que fue a museo no pagó entrada. 
Osea sugerir una semana de los museos como la noche de los museos con entrada gratis para q la gente pueda ir.

Se pueden meter los modelos acá también, poner en Eje X la edad, en Eje Y NSEpuntaje (más bajo, menor clase social, más álto el puntaje mayor clase social) y buscar modelar ahí y ajustar por cantidad de autos, si concurrió o no a museo, por región, etc. No estoy seguro q sale de eso.

\end{frame}

\end{document}
